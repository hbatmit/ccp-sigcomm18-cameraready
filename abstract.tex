\begin{abstract}

This paper describes the implementation and evaluation of a Congestion Control Plane (CCP), a way to implement congestion control functions at the sender by removing them from the datapath. 
With CCP, each datapath---such as the Linux Kernel TCP, UDP-based QUIC, or kernel-bypass transports like mTCP-on-DPDK---summarizes information about the round-trip times, packet receptions, losses, ECN, etc. via a well-defined interface to congestion-control algorithms running in CCP. 
The algorithms use this information to control the datapath's congestion window or pacing rate. We show that CCP enables users to write congestion control algorithms once and run them on multiple datapaths. 
CCP improves both the pace of development and ease of maintenance of congestion control algorithms by providing better, modular abstractions, and supports capabilities such as the Congestion Manager, all with one-time changes to datapaths. 
CCP also enables new capabilities, such as Copa in Linux TCP, several algorithms running on QUIC and mTCP/DPDK, and the use of signal processing algorithms to detect whether cross-traffic is ACK-clocked.
Experiments with our user-level Linux CCP implementation show that CCP algorithms behave similarly to kernel algorithms, and incur modest CPU overhead of a few percent.
\end{abstract}
